
% preamble
\documentclass[11pt,reqno,oneside,a4paper]{article}
\usepackage[a4paper,includeheadfoot,left=25mm,right=25mm,top=00mm,bottom=20mm,headheight=20mm]{geometry}
%%%% DO NOT EDIT THIS FILE

% Standard packages
\usepackage{amssymb,amsmath,amsthm}
\usepackage{xcolor,graphicx}
\usepackage{verbatim}
\usepackage{mathtools}
\usepackage{hyperref}
% Layout of headers & footers
\usepackage{titling}
\usepackage{fancyhdr}
\newcommand{\runningtitle}{Running Title}
\pagestyle{fancy} \lhead{{\theauthor}} \chead{} \rhead{{\runningtitle}} \lfoot{} \cfoot{\thepage} \rfoot{}

% Hyphenation
\hyphenation{non-zero}

% Theorem definitions in the amsthm standard
\newtheorem{thm}{Theorem}
\newtheorem{lem}[thm]{Lemma}
\newtheorem{sublem}[thm]{Sublemma}
\newtheorem{prop}[thm]{Proposition}
\newtheorem{cor}[thm]{Corollary}
\newtheorem{conc}[thm]{Conclusion}
\newtheorem{conj}[thm]{Conjecture}
\theoremstyle{definition}
\newtheorem{defn}[thm]{Definition}
\newtheorem{cond}[thm]{Condition}
\newtheorem{asm}[thm]{Assumption}
\newtheorem{ntn}[thm]{Notation}
\newtheorem{prob}[thm]{Problem}
\theoremstyle{remark}
\newtheorem{rmk}[thm]{Remark}
\newtheorem{eg}[thm]{Example}
\newtheorem*{hint}{Hint}

%% Mathmode shortcuts
% Number sets
\newcommand{\NN}{\mathbb N}              % The set of naturals
\newcommand{\NNzero}{\NN_0}              % The set of naturals including zero
\newcommand{\NNone}{\NN}                 % The set of naturals excluding zero
\newcommand{\ZZ}{\mathbb Z}              % The set of integers
\newcommand{\QQ}{\mathbb Q}              % The set of rationals
\newcommand{\RR}{\mathbb R}              % The set of reals
\newcommand{\CC}{\mathbb C}              % The set of complex numbers
\newcommand{\KK}{\mathbb K}              % An arbitrary field
% Modern typesetting for the real and imaginary parts of a complex number
\renewcommand{\Re}{\operatorname*{Re}} \renewcommand{\Im}{\operatorname*{Im}}
% Upright d for derivatives
\newcommand{\D}{\ensuremath{\,\mathrm{d}}}
% Upright i for imaginary unit
\newcommand{\ri}{\ensuremath{\mathrm{i}}}
% Upright e for exponentials
\newcommand{\re}{\ensuremath{\mathrm{e}}}
% abbreviation for \lambda
\newcommand{\la}{\ensuremath{\lambda}}
% Make epsilons look more different from the element symbol
\renewcommand{\epsilon}{\varepsilon}
% Always use slanted forms of \leq, \geq
\renewcommand{\geq}{\geqslant}
\renewcommand{\leq}{\leqslant}
% Shorthand for "if and only if" symbol
\newcommand{\Iff}{\ensuremath{\Leftrightarrow}}
% Make bold symbols for vectors
\providecommand{\BVec}[1]{\mathbf{#1}}
% Hyperbolic functions
\providecommand{\sech}{\operatorname{sech}}
\providecommand{\csch}{\operatorname{csch}}
\providecommand{\ctnh}{\operatorname{ctnh}}
% sinc function
\providecommand{\sinc}{\operatorname{sinc}}
% closure of a set
\providecommand{\clos}{\operatorname{clos}}
% The absolute value of a real number or modulus of a complex number, with automatically scaling delimiters
\newcommand{\abs}[1]{\left\lvert#1\right\rvert}
\newcommand{\sgn}{\operatorname{sgn}}

% add two sub and superscripts with a space between them
\newcommand{\Mspacer}{\;} %Spacer for below Matrix display functions
\newcommand{\M}[3]{#1_{#2\Mspacer#3}} %Print a symbol with two subscripts eg a matrix entry
\newcommand{\Msup}[4]{#1_{#2\Mspacer#3}^{#4}} %Print a symbol with two subscripts and a superscript eg a matrix entry
\newcommand{\Msups}[5]{#1_{#2\Mspacer#3}^{#4\Mspacer#5}} %Print a symbol with two subscripts and two superscripts eg a matrix entry
\newcommand{\MAll}[7]{\prescript{#1}{#2}{#3}_{#4\Mspacer#5}^{#6\Mspacer#7}} %Print a symbol with two subscripts and two superscripts eg a matrix entry

% Make really wide hat for Fourier transforms applied to large functions
\usepackage{scalerel}
\usepackage{stackengine}
\stackMath
\newcommand\reallywidecheck[1]{%
\savestack{\tmpbox}{\stretchto{%
  \scaleto{%
    \scalerel*[\widthof{\ensuremath{#1}}]{\kern-.6pt\bigwedge\kern-.6pt}%
    {\rule[-\textheight/2]{1ex}{\textheight}}%WIDTH-LIMITED BIG WEDGE
  }{\textheight}%
}{0.5ex}}%
\stackon[1pt]{#1}{\scalebox{-1}{\tmpbox}}%
}
\providecommand{\widecheck}{\reallywidecheck}

\newcommand\reallywidehat[1]{%
\savestack{\tmpbox}{\stretchto{%
  \scaleto{%
    \scalerel*[\widthof{\ensuremath{#1}}]{\kern-.6pt\bigwedge\kern-.6pt}%
    {\rule[-\textheight/2]{1ex}{\textheight}}%WIDTH-LIMITED BIG WEDGE
  }{\textheight}%
}{0.5ex}}%
\stackon[1pt]{#1}{\tmpbox}%
}


%% Acknowledgements
\newcommand{\AckYNCSRP}[1]{#1 gratefully acknowledges support from Yale-NUS College summer research programme.}
\newcommand{\AckYNCProj}[1]{#1 gratefully acknowledges support from Yale-NUS College project B grant IG18-PRB102.}
\newcommand{\AckYNCWorkshop}[1]{#1 gratefully acknowledges support from Yale-NUS College workshop grant IG18-CW003.}
\newcommand{\AckNICA}[1]{#1 would like to thank the Isaac Newton Institute for Mathematical Sciences for support and hospitality during programme \emph{Complex analysis: techniques, applications and computations}, when work on this paper was undertaken. This work was supported by EPSRC Grant Number EP/R014604/1.}
\newcommand{\AckSMRIIVP}[1]{#1 would like to thank the Sydney Mathematics Research Institute for support and hospitality under the International Visitor Programme.}
 % Standard packages, page layout, theorem environments, macros, etc
% This file contains macros specific to the project.
% You are welcome to add your own macros, but please avoid deleting those written by others.

% Asymptotic notation
\newcommand{\bigoh}{\mathcal{O}}
\newcommand{\lindecayla}{\bigoh\left(\abs{\la}^{-1}\right)}
 % Macros specific to this project.
\author{Juwon Lee}
\title{Notes on the Argument Principle}
\renewcommand{\runningtitle}{Notes on the Argument Principle}
\date{\today}

\begin{document}
\maketitle
\thispagestyle{fancy}

\begin{abstract}
In complex analysis, the argument principle (or Cauchy's argument principle) relates the difference between the number of zeros and poles of a meromorphic function to a contour integral of the function's logarithmic derivative. 
The aim of this paper is to explain the concept, but not necessarily the proofs, to someone who has completed the same MCS modules as I did.
\end{abstract}

\section{Argument Principle}
\subsection{Introduction}
The argument principle connects the winding number of a curve with the number of zeros and poles inside the curve. This principle is useful for mathematical applications where we want to locate the zeros and poles. 

\begin{asm}
Note that this research project will be looking specifically at holomorphic functions, hence the number of poles will be $0.$ Therefore, the argument principle will be simplified to exclusively locate zeros and not poles.
\end{asm}

\subsection{Argument Principle}
The argument principle relates the difference between the number of zeros of a holomorphic function to a contour integral of the function's logarithmic derivative. 
The contour integral $\oint\limits_{C} \frac{f'(z)}{f(z)}\mathrm{d}z $ can be interpreted as $2\pi i$ times the winding number of the path $f(C)$ around the origin. 
That is, it is $i$ times the total change in the argument of $f(z)$ as $z$ travels around $C$, explaining the name of the theorem. 
\begin{thm}{Argument Principle.}
If $f(z)$ is a holomorphic function inside and on some closed contour $C$, and $f$ has no zeros on $C$, then 
$$ \frac{1}{2\pi i} \oint\limits_{C} \frac{f'(z)}{f(z)}\mathrm{d}z = Z$$
where $Z$ denote the number of zeros of $f(z)$ inside the contour $C$, with each zero counted as many times as its multiplicity and order. 
\end{thm}

For better understanding, this paper provides the definition of winding numbers, holomorphic functions, zeros, and contour integrals.


\subsection{Supplementary Definitions}
% Winding numbers 
\begin{defn}{Winding numbers.}
The winding number of a closed curve in the plane around a given point is an integer representing the total number of times that curve travels counterclockwise around the point. 
\end{defn}

Let's take a look at an example for better understanding of winding numbers and the use of the argument principle.
\begin{prob}{Winding numbers.}
Let $f(z) = z^2+z$. Find the winding number of $f \circ \gamma$ around $0$ for 
$\gamma_1 =$ circle of radius $2,$ 
$\gamma_2 =$ circle of radius $\frac{1}{2},$ and
$\gamma_3 =$ circle of radius $1$. 

\begin{enumerate}
$f(z) = z(z+1)$ has zeros at $0$ and $-1$.
\item For $\gamma_1,$ there are two zeros of both $0$ and $-1.$
\item For $\gamma_2,$ there is one zero of $0$.
\item For $\gamma_3,$ there is one zero of $0.$ $-1$ is not inside the circle, therefore it does not count.
\end{enumerate}
\end{prob}

% Holomorphic function
\begin{defn}{Holomorphic function.}
A holomorphic function is a complex-valued function of one or more complex variables that is, at every point of its domain, complex differentiable in a neighborhood of the point. 
\end{defn}

% Meromorphic function definition
\begin{defn}{Meromorphic function.}
A meromorphic function on an open subset $D$ of the complex plane is a function that is holomorphic on all of $D$ except for a set of isolated points, which are poles of the function. 
\end{defn}

% Zero definition
\begin{defn}{Zeros.}
Zeros of holomorphic functions are points $z$ where $$f(z) = 0.$$
\end{defn}

% Contour integral
\begin{defn}{Contour integral.}
Contour integration is a method of evaluating certain integrals along the paths in the complex plane. 
For continuous functions in the complex plane, the contour integral can be defined in analogy to the line integral by first defining the integral along a directed smooth curve in terms of an integral over a real valued parameter.
\end{defn}

\begin{ntn}{Contour integral.}
Contour integrals are denoted by $\oint$.
\end{ntn}



\end{document}