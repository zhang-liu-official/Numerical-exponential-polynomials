% preamble
\documentclass[11pt,reqno,oneside,a4paper]{article}
\usepackage[a4paper,includeheadfoot,left=25mm,right=25mm,top=00mm,bottom=20mm,headheight=20mm]{geometry}
\input{../texhead-main} % Standard packages, page layout, theorem environments, macros, etc
\input{../texhead-project} % Macros specific to this project.
\author{Wang Yanhua}
\title{Asymptotic zero locus}
\renewcommand{\runningtitle}{Asymptotic zero locus}
\date{\today}

\begin{document}
\maketitle
\thispagestyle{fancy}

\section{Hypothesis}

For $x>0$, as $x \to \infty, f(x)=g(x)$ approximately when $f(x) = 0$, ie when $x$ is an integer multiple of $\pi$.

\section{Qualitative asymptotics}

As $x$ gets larger, the approximation gets better. This is because $g(x)$ tends to zero. 

\section{Quantitative asymptotics}

First, we recall the Maclaurin series representations for $\cos, \sin,$ and exponential functions.
\begin{align*}
\cos x &= \sum_{n=0}^{\infty} \frac{(-1)^nx^{2n}}{(2n)!}. \\
\sin x &= \sum_{n=0}^{\infty} \frac{(-1)^nx^{2n+1}}{(2n+1)!}. \\
e^x &= \sum_{n=0}^{\infty} \frac{x^n}{n!}.
\intertext{Hence,}
e^{-x}&=\sin x \iff \\
1 - x + \frac{x^2}{2!}-\frac{x^3}{3!} + \cdots &= x - \frac{x^3}{3!} + \frac{x^5}{5!} -\frac{x^7}{7!} + \cdots \iff \\
1 - 2x +  \frac{x^2}{2!} +  \frac{x^4}{4!} -2 \frac{x^5}{5!} &+  \frac{x^6}{6!} + \cdots = 0 \iff \\
(1+\frac{x^2}{2!} + \frac{x^4}{4!} +  \frac{x^6}{6!} + \cdots) &- 2(x +  \frac{x^5}{5!}  +  \frac{x^9}{9!} +  \frac{x^{13}}{13!} + \cdots) = 0.
\intertext{At the kth positive solution, the bound on the error is }
\end{align*}

\end{document}