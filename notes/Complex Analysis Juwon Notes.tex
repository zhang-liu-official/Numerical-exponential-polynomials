%---------DO NOT EDIT THE FOLLOWING INDENTED SECTION
	% Preamble
	\documentclass{article}
	\usepackage[a4paper,includeheadfoot,left=35mm,right=35mm,top=00mm,bottom=30mm,headheight=40mm]{geometry} %sets up the margins
	\usepackage{amssymb,amsmath,amsthm}
	\usepackage{xcolor,graphicx}
	\usepackage{hyperref}

	\usepackage{titling}
	\usepackage{fancyhdr}
	\pagestyle{fancy}
	\lhead{{\theauthor}}
	\chead{}
	\rhead{}
	\lfoot{}
	\cfoot{\thepage}
	\rfoot{}


	\title{Complex Analysis Notes}
	\author{Juwon Lee}
	\date{\today}   

	\theoremstyle{definition}
	\newtheorem{definition}{Definition}[section]

	\theoremstyle{remark}
	\newtheorem*{remark}{Remark}

	
	\newcounter{subdefinition}[definition]
	\renewcommand{\thesubdefinition}{\thedefinition.\arabic{subdefinition}}
	\newenvironment{subdefinition}{
	\refstepcounter{subdefinition}
	\par\noindent
	\textbf{\upshape Definition \thesubdefinition}%
	}{}
	
	\newtheorem{theorem}{Theorem}[section]
	\newtheorem{corollary}{Corollary}[theorem]
	\newtheorem{lemma}[theorem]{Lemma}
	
	\usepackage{mathtools}

	\DeclarePairedDelimiter\abs{\lvert}{\rvert}%
	\DeclarePairedDelimiter\norm{\lVert}{\rVert}%

%----------------------------------


\begin{document}
\maketitle
\thispagestyle{fancy}
%-----------THIS IS WHERE THE MAIN DOCUMENT BEGINS.

The goal of this document is to revise complex numbers, complex functions and analytic functions before diving into the argument principle. In addition, I aim to review my skills for LaTeX documentation. 


\section{Complex Numbers and Functions}

Cardan discovered the usefulness of imaginary numbers in computation, Euler introduced the symbol i, and Gauss recognized the importance of complex numbers. \par
\subsection{Forms of Complex Numbers}
\begin{enumerate}
\item Algebraic form: $x + iy$ where $x,y$ are real numbers. x is the real part, and y is the imaginary part
\item Geometric form of vectors: $(x,y)$
\item Polar form: $z =r(\cos\theta + i\sin\theta)$, where $$r = \sqrt{x^2+y^2} > 0, \cos \theta = \frac{x}{r}, sin \theta = \frac{y}{r}.$$ The argument $z$ is not defined when $z=0$ or equivalently when $r=0.$
\end{enumerate}




\subsection{Complex Functions}

\begin{definition}[Complex-valued function] $f$ of a complex variable is a relation that assigns to each complex number $z$ in a set $S$ a unique complex number $f(z).$ \par
\end{definition}
\begin{definition}
The set $S$ is a subset of the complex numbers and is called the domain of definition of $f.$ \par
\end{definition}

Visualization of complex-valued functions require four dimensions: two for variable $z$ and two for the values $w=f(z).$ In reality, we use two planes, $z$-plane and $w$-plane and view the function as a mapping from a subset of one plane to the other.
\par

\subsubsection{Linear transformations }
Linear transformations can be thought of in terms of a dilation, a rotation, and a translation, which map regions to geometrically similar regions.


\section{Analytic Functions}

Most of the theory of analytic functions is due to Augustin-Louis Cauchy.
\par
\begin{enumerate}
\item Defined the derivative and integral of complex functions
\item Defined the notion of limit for functions and gave rigorous definitions of continuity and differentiability for real-valued functions
\item Developed groundwork for the theory of definite integrals and series 
\item Established theoretical aspects of complex analysis with great attention to rigorous mathematical proof which characterizes pure mathematics
\end{enumerate}


\begin{definition}{(Neighborhoods)}
Let $r>0$ be a positive real number and $z_0$ a point in the plane. The $r$-neighborhood of $z_0$ is the set of all complex numbers $z$ satisfying $\abs{z-z_0} < r.$ We denote this set by $B_r(z_0).$
\end{definition}

\begin{subdefinition}{. (Deleted/Punctured Neighborhood)}
$$B'_r(z_0) = {z: 0 < \abs{z-z_0} < r}.$$
\end{subdefinition}

\begin{subdefinition}
Let $S$ be a subset of $\mathbb{C}$.  
\begin{enumerate}
\item Interior point: $z_0$ is an interior point of $S$ if we can find a neighborhood of $z_0$ that is wholly contained in $S$.
\item Boundary point: $z$ in the complex plane is called boundary point of $S$ if every neighborhood of $z$ contains at least one point in $S$ and at least one point not in $S$. 
\item Boundary: the set of all boundary points of $S$ is called the boundary of $S$.
\end{enumerate}
\end{subdefinition}

\begin{subdefinition}
A subset $S$ of the complex numbers is called open if every point in $S$ is an interior point of $S$. An r-neighborhood, $B_r(z_0)$ is an open disk of radius $r$ centered at $z_0.$ Sets that contain all of their boundary points are called closed. For example, ${z: \abs{z-z_0} \leq r}$ is a closed disk. The smallest closed set that contains a set $A$ is called the closure of $A$.  
\end{subdefinition}

\begin{definition}{(Complex Derivative)}
Let $f$ be defined on an open subset $U$ of $\mathbb{C}$ and let $z_0 \in U.$ We say that $f$ has a complex derivative at the point $z_0$ if the limit $$\lim{z\to z_0}\frac{f(z)-f(z_0)}{z-z_0}$$ exists. This is called the complex derivative of $f$ at $z_0$ and is denoted by $f'(z_0)$.
\end{definition}
We say that $f$ is analytic on $U$ if it has a complex derivative at every point in $U$.

\begin{subdefinition}
An analytic function defined on the complex plane is said to be entire. 
\end{subdefinition}
\begin{lemma}
If $c$ is a constant, $f(z) = cz \Rightarrow f'(z) = c.$ 
\end{lemma}
\begin{theorem}
An analytic function defined on an open subset of the complex plane is continuous.
\end{theorem}
\begin{lemma}
Discontinuous function at $z_0$ does not have a complex derivative at $z_0.$
\end{lemma}

\begin{theorem}[Properties of Analytic Functions]
Suppose that $f$ and $g$ are analytic functions on an open subset $U$ of the complex plane and let $c_1,c_2$ be complex constants. Then,
\begin{enumerate}
\item $c_1f+c_2g$ and $fg$ are analytic on $U$ and for all $z \in U$.
\item The function $fg$ is analytic on $U$ and for all $z \in U.$
\item The function $\frac{f}{g}$ is analytic on $W = U \setminus {w\in U : g(w) = 0}$ and for all $z\in W$, 
$$ (\frac{f}{g})'(z) = \frac{f'(z)g(z)-f(z)g'(z)}{(g(z))^2}$$
\end{enumerate}
\end{theorem}

\begin{theorem}[Cauchy-Riemann Equations]
Let $U$ be an open subset of $\mathbb{R}^2$ and let $u,v$ be real-valued functions defined on $U.$ Then the complex-valued function $f(x+iy) = u(x,y) + iv(x,y)$ is analytic on $U$ if and only if $u,v$ are differentiable functions on $U$ and satisfy $$u_x=v_y, u_y = -v_x$$ for all points in $U$. If this is the case, then for all $(x,y)\in U$, we have $$f'(x+iy) = u_x(x,y)+iv_x(x,y) \text{or} f'(x_iy) = v_y(x,y) - iu_y(x,y).$$
\end{theorem}






\end{document}

