% preamble
\documentclass[11pt,reqno,oneside,a4paper]{article}
\usepackage[a4paper,includeheadfoot,left=25mm,right=25mm,top=00mm,bottom=20mm,headheight=20mm]{geometry}
%%%% DO NOT EDIT THIS FILE

% Standard packages
\usepackage{amssymb,amsmath,amsthm}
\usepackage{xcolor,graphicx}
\usepackage{verbatim}
\usepackage{mathtools}
\usepackage{hyperref}
% Layout of headers & footers
\usepackage{titling}
\usepackage{fancyhdr}
\newcommand{\runningtitle}{Running Title}
\pagestyle{fancy} \lhead{{\theauthor}} \chead{} \rhead{{\runningtitle}} \lfoot{} \cfoot{\thepage} \rfoot{}

% Hyphenation
\hyphenation{non-zero}

% Theorem definitions in the amsthm standard
\newtheorem{thm}{Theorem}
\newtheorem{lem}[thm]{Lemma}
\newtheorem{sublem}[thm]{Sublemma}
\newtheorem{prop}[thm]{Proposition}
\newtheorem{cor}[thm]{Corollary}
\newtheorem{conc}[thm]{Conclusion}
\newtheorem{conj}[thm]{Conjecture}
\theoremstyle{definition}
\newtheorem{defn}[thm]{Definition}
\newtheorem{cond}[thm]{Condition}
\newtheorem{asm}[thm]{Assumption}
\newtheorem{ntn}[thm]{Notation}
\newtheorem{prob}[thm]{Problem}
\theoremstyle{remark}
\newtheorem{rmk}[thm]{Remark}
\newtheorem{eg}[thm]{Example}
\newtheorem*{hint}{Hint}

%% Mathmode shortcuts
% Number sets
\newcommand{\NN}{\mathbb N}              % The set of naturals
\newcommand{\NNzero}{\NN_0}              % The set of naturals including zero
\newcommand{\NNone}{\NN}                 % The set of naturals excluding zero
\newcommand{\ZZ}{\mathbb Z}              % The set of integers
\newcommand{\QQ}{\mathbb Q}              % The set of rationals
\newcommand{\RR}{\mathbb R}              % The set of reals
\newcommand{\CC}{\mathbb C}              % The set of complex numbers
\newcommand{\KK}{\mathbb K}              % An arbitrary field
% Modern typesetting for the real and imaginary parts of a complex number
\renewcommand{\Re}{\operatorname*{Re}} \renewcommand{\Im}{\operatorname*{Im}}
% Upright d for derivatives
\newcommand{\D}{\ensuremath{\,\mathrm{d}}}
% Upright i for imaginary unit
\newcommand{\ri}{\ensuremath{\mathrm{i}}}
% Upright e for exponentials
\newcommand{\re}{\ensuremath{\mathrm{e}}}
% abbreviation for \lambda
\newcommand{\la}{\ensuremath{\lambda}}
% Make epsilons look more different from the element symbol
\renewcommand{\epsilon}{\varepsilon}
% Always use slanted forms of \leq, \geq
\renewcommand{\geq}{\geqslant}
\renewcommand{\leq}{\leqslant}
% Shorthand for "if and only if" symbol
\newcommand{\Iff}{\ensuremath{\Leftrightarrow}}
% Make bold symbols for vectors
\providecommand{\BVec}[1]{\mathbf{#1}}
% Hyperbolic functions
\providecommand{\sech}{\operatorname{sech}}
\providecommand{\csch}{\operatorname{csch}}
\providecommand{\ctnh}{\operatorname{ctnh}}
% sinc function
\providecommand{\sinc}{\operatorname{sinc}}
% closure of a set
\providecommand{\clos}{\operatorname{clos}}
% The absolute value of a real number or modulus of a complex number, with automatically scaling delimiters
\newcommand{\abs}[1]{\left\lvert#1\right\rvert}
\newcommand{\sgn}{\operatorname{sgn}}

% add two sub and superscripts with a space between them
\newcommand{\Mspacer}{\;} %Spacer for below Matrix display functions
\newcommand{\M}[3]{#1_{#2\Mspacer#3}} %Print a symbol with two subscripts eg a matrix entry
\newcommand{\Msup}[4]{#1_{#2\Mspacer#3}^{#4}} %Print a symbol with two subscripts and a superscript eg a matrix entry
\newcommand{\Msups}[5]{#1_{#2\Mspacer#3}^{#4\Mspacer#5}} %Print a symbol with two subscripts and two superscripts eg a matrix entry
\newcommand{\MAll}[7]{\prescript{#1}{#2}{#3}_{#4\Mspacer#5}^{#6\Mspacer#7}} %Print a symbol with two subscripts and two superscripts eg a matrix entry

% Make really wide hat for Fourier transforms applied to large functions
\usepackage{scalerel}
\usepackage{stackengine}
\stackMath
\newcommand\reallywidecheck[1]{%
\savestack{\tmpbox}{\stretchto{%
  \scaleto{%
    \scalerel*[\widthof{\ensuremath{#1}}]{\kern-.6pt\bigwedge\kern-.6pt}%
    {\rule[-\textheight/2]{1ex}{\textheight}}%WIDTH-LIMITED BIG WEDGE
  }{\textheight}%
}{0.5ex}}%
\stackon[1pt]{#1}{\scalebox{-1}{\tmpbox}}%
}
\providecommand{\widecheck}{\reallywidecheck}

\newcommand\reallywidehat[1]{%
\savestack{\tmpbox}{\stretchto{%
  \scaleto{%
    \scalerel*[\widthof{\ensuremath{#1}}]{\kern-.6pt\bigwedge\kern-.6pt}%
    {\rule[-\textheight/2]{1ex}{\textheight}}%WIDTH-LIMITED BIG WEDGE
  }{\textheight}%
}{0.5ex}}%
\stackon[1pt]{#1}{\tmpbox}%
}


%% Acknowledgements
\newcommand{\AckYNCSRP}[1]{#1 gratefully acknowledges support from Yale-NUS College summer research programme.}
\newcommand{\AckYNCProj}[1]{#1 gratefully acknowledges support from Yale-NUS College project B grant IG18-PRB102.}
\newcommand{\AckYNCWorkshop}[1]{#1 gratefully acknowledges support from Yale-NUS College workshop grant IG18-CW003.}
\newcommand{\AckNICA}[1]{#1 would like to thank the Isaac Newton Institute for Mathematical Sciences for support and hospitality during programme \emph{Complex analysis: techniques, applications and computations}, when work on this paper was undertaken. This work was supported by EPSRC Grant Number EP/R014604/1.}
\newcommand{\AckSMRIIVP}[1]{#1 would like to thank the Sydney Mathematics Research Institute for support and hospitality under the International Visitor Programme.}
 % Standard packages, page layout, theorem environments, macros, etc
% This file contains macros specific to the project.
% You are welcome to add your own macros, but please avoid deleting those written by others.

% Asymptotic notation
\newcommand{\bigoh}{\mathcal{O}}
\newcommand{\lindecayla}{\bigoh\left(\abs{\la}^{-1}\right)}
 % Macros specific to this project.
\author{Juwon Lee}
\title{Implementation Overview of Locating Zeros of Exponential Polynomials}
\renewcommand{\runningtitle}{Code Report}
\date{\today}

\begin{document}
\maketitle
\thispagestyle{fancy}
%----------------------------------

\begin{abstract}
	The argument principle connects the winding number of a curve with the number of zeros inside the curve. 
	This principle has been applied into coding in the programming language julia to locate the zeros of big complex functions.
	This document will give a high-level overview of the algorithm that spots the zeros of the exponential polynomials.
	I start by explaining the argument principle and the reason for using julia before explaining the code.
\end{abstract}

\section{The Argument Principle}
	
\begin{defn}{Holomorphic function.}
A holomorphic function is a complex-valued function of one or more complex variables that is, at every point of its domain, complex differentiable in a neighborhood of the point. 
\end{defn}

Note that this research project will be looking specifically at holomorphic functions, hence the number of poles will be $0.$ 
Therefore, the argument principle will be simplified to exclusively locate zeros and not poles.

\begin{thm}
The argument principle relates the difference between the number of zeros of a holomorphic function to a contour integral of the function's logarithmic derivative. 
The contour integral $\oint\limits_{C} \frac{f'(z)}{f(z)}\mathrm{d}z $ can be interpreted as $2\pi i$ times the winding number of the path $f(C)$ around the origin. 
\end{thm}

That is, it is $i$ times the total change in the argument of $f(z)$ as $z$ travels around $C$, explaining the name of the theorem. 
The goal of the implementation is to spot the zeros of the analytic function within the input area, given in the shape of a rectangle.
The next section explains the data structures and functions built on the logic of the argument principle.

\section{Why Julia?}
Julia is a high-level, open-source, dynamically-typed programming language developed in 2012.
It was designed to excel at scientific and numerical computing allowing faster compilation than the existing languages such as C, or Python.
This is possible due to the Just-In-Time (JIT) compilation method, where every statement is run using compiled functions which are either compiled right before they are used, or cached from before.

Once the size of the exponential polynomials outgrow one that can be solved by hand, julia will allow for an efficient and fast way to locate the zeros of the exponential polynomial.


\section{High-level Overview of the Implementation}
\subsection{Data Structure}
	The first struct Rect stores two coordinates of the input rectangle. 
	The type of the two inputs are tuples of floats.
	The reason I take two coordinates instead of four coordinates of the rectangle is to ensure that the inputs lead to rectangles instead of error checking that four coordinates are in the shape of a rectangle. 
	
	The second struct Step stores the information of the height, width, and stepping points of the rectangle stored in Rect, all of them of type float.
	The stepping points called step decides the gap between the point in the rectangle to evaluate and the next. 
	This is adjusted accordingly in the function parseInput, explained in the next subsection. 
	
\subsection{Functions}
	\begin{enumerate}
		\item \emph{parseInput}: 
		This function takes in the bottom left coordinate and the top right coordinate of the rectangle and calculates the height and width of the rectangle. 
		If the step of the loop is too big, it may miss certain jumps and not detect the correct number of zeros of the analytic function.
		To avoid this, I tested the function with unit tests and set the parameter of $n$ to $2000.$ 
		
		\item \emph{argBox}:
		This function evaluates the arguments along the borders of the rectangle in the analytic function. 
		It starts the evaluation with the bottom left coordinate, and increments the steps by the floating point stored in the struct Step. 
		In the same way, the other three borders of the rectangle are evaluated and stored into the resulting array. 
		To make sure the argument evaluation was closed on the rectangle, at the end of the code, I added the argument evaluation of the starting coordinate which in this case is the bottom left coordinate.
		
		\item \emph{countJump}: 
		This function takes in the array of arguments that were evaluated using the function argBox and counts the number of jumps.
		I set the parameter to be $-4$ which is slightly less than $-2\pi$ at first because the start of a point may not be exactly at $\pi$ or $-\pi.$ 
		
		\item \emph{divideRec}:
		This recursive function subdivides the input rectangle into four sub-rectangles.
		It runs the recursion until I obtain the bottom left coordinate of each of the zeros. 
		
		
	\end{enumerate}

\section{Testing}
	
Testing is an essential part of the algorithm for two purposes. 
First, in the beginning stage of drafting the code, writing unit tests help the programmer to understand the problem they are trying to solve. 
In this project, I was able to better comprehend how to take in the input rectangle, what the output of the countJump function should be by writing the unit tests first.

Second, testing is an essential part of finding corner cases. 
In this project, the second unit test $t2$ could not locate the zero on the border of the rectangle. 

	
	
	
	
	
	
	
	
	
	
	
	
	
	
	
	
	
	
	

\end{document}