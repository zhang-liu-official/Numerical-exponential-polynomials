% Preamble
\documentclass[11pt,reqno,oneside,a4paper]{article}
\usepackage[a4paper,includeheadfoot,left=25mm,right=25mm,top=00mm,bottom=20mm,headheight=20mm]{geometry}
\input{../texhead-main} % Standard packages, page layout, theorem environments, macros, etc
\input{../texhead-project} % Macros specific to this project.
\author{Wang Yanhua}
\title{Quantitative Asymptotics}
\renewcommand{\runningtitle}{Quantitative Asymptotics}
\date{\today}

\begin{document}
\maketitle
\thispagestyle{fancy}

\section*{Error bounds}

The 1st degree Taylor polynomial for $\sin (x)$ at each $x_k$ is 
$$\sin(k\pi) + \cos(k\pi) (x-k\pi).$$
The error bound for it is 
$$ \frac{\max(-\sin(c))}{2}(x_k - k\pi)^2,$$ 
where $c$ is between $k\pi$ and $x_k$. The absolute value of $-\sin(x_k)$ is always larger than $\sin(k\pi) =0$, hence $c = x_k$. Hence, the error bound is 
$$\abs{-\frac{\sin(x_k)}{2}(x_k - k\pi)^2}.$$

The 1st degree Taylor polynomial for  $e^{-x}$ at each $x_k$ is
$$e^{-k\pi} - e^{-k\pi}(x-k\pi). $$
The error bound for it is $$ \frac{\max(e^{-c})}{2}(x_k - k\pi)^2,$$ where $c$ is between $k\pi$ and $x_k$. $e^{-k\pi}$ is larger when $k$ is an even number, while $e^{-x_k}$ is larger when $k$ is an odd number.

The error bounds on the linear approximation solutions for $\sin (x) - e^{-x} = 0$ is hence $$\abs{-\frac{\sin(x_k)}{2}(x_k - k\pi)^2}+ \frac{e^{-k\pi}}{2}(x_k - k\pi)^2 $$ when $k$ is even, and  $$\abs{-\frac{\sin(x_k)}{2}(x_k - k\pi)^2} +  \frac{e^{-x_k}}{2}(x_k - k\pi)^2$$ when $k$ is odd.

\end{document}
